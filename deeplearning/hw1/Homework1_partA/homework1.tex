\documentclass[12pt]{article}%
\usepackage{amsfonts}
\usepackage{fancyhdr}
\usepackage{comment}
\usepackage[a4paper, top=2.5cm, bottom=2.5cm, left=2.2cm, right=2.2cm]%
{geometry}
\usepackage{times}
\usepackage{amsmath}
\usepackage{changepage}
\usepackage{amssymb}
\usepackage{graphicx}
\usepackage{url}%
\setcounter{MaxMatrixCols}{30}
\newtheorem{theorem}{Theorem}
\newtheorem{acknowledgement}[theorem]{Acknowledgement}
\newtheorem{algorithm}[theorem]{Algorithm}
\newtheorem{axiom}{Axiom}
\newtheorem{case}[theorem]{Case}
\newtheorem{claim}[theorem]{Claim}
\newtheorem{conclusion}[theorem]{Conclusion}
\newtheorem{condition}[theorem]{Condition}
\newtheorem{conjecture}[theorem]{Conjecture}
\newtheorem{corollary}[theorem]{Corollary}
\newtheorem{criterion}[theorem]{Criterion}
\newtheorem{definition}[theorem]{Definition}
\newtheorem{example}[theorem]{Example}
\newtheorem{exercise}[theorem]{Exercise}
\newtheorem{lemma}[theorem]{Lemma}
\newtheorem{notation}[theorem]{Notation}
\newtheorem{problem}[theorem]{Problem}
\newtheorem{proposition}[theorem]{Proposition}
\newtheorem{remark}[theorem]{Remark}
\newtheorem{solution}[theorem]{Solution}
\newtheorem{summary}[theorem]{Summary}
\newenvironment{proof}[1][Proof]{\textbf{#1.} }{\ \rule{0.5em}{0.5em}}

\newcommand{\Q}{\mathbb{Q}}
\newcommand{\R}{\mathbb{R}}
\newcommand{\C}{\mathbb{C}}
\newcommand{\Z}{\mathbb{Z}}

\begin{document}

\title{CS280 Fall 2021 Assignment 1 \\ Part A}
\author{ML Background}
\maketitle

\paragraph{Name:}

\paragraph{Student ID:}

\newpage


\subsubsection*{1. MLE (5 points)}
Given a dataset $\mathcal{D} = \{x_1,\cdots, x_n\}$. Let $p_{emp}(x)$ be the empirical distribution, i.e., $p_{emp}(x)=\frac{1}{n}\sum_{i=1}^n\delta(x,x_i) $ where $\delta(x,a)$ is the Dirac delta function\footnote{\url{https://en.wikipedia.org/wiki/Dirac_delta_function}} centered at $a$. Assume $q(x|\theta)$ be some probabilistic model.   
\begin{itemize}
	\item Show that $\arg\min_q KL(p_{emp}||q)$ is obtained by $q(x)=q(x;\hat{\theta})$, where $\hat{\theta}$ is the Maximum Likelihood Estimator and $KL(p||q)=\int p(x)(\log p(x)- \log q(x))dx$ is the KL divergence.
\end{itemize}





\newpage




%\subsubsection*{3. Gaussian Distributions (10 points)}
%Let $X\sim N(0,1)$ and $Y=WX$, where $p(W=-1)=p(W=1)=0.5$. It is clear that $X$ and $Y$ are not independent since $Y$ is a function of $X$. 
%\begin{itemize}
%	\item Show $Y\sim N(0,1)$
%	\item Show $cov[X,Y]=0$. hint: $cov[X,Y]=E[XY]-E[X]E[Y]$ and $E[XY]=E[E[XY|W]]$
%\end{itemize}
%Therefore, $X$ and $Y$ are uncorrelated and Gaussian, but they are dependent. Why?
\subsubsection*{2. Gradient descent for fitting GMM (10 points)}
Consider the Gaussian mixture model
\[p(\mathbf{x}|\theta)=\sum_{k=1}^K \pi_k \mathcal{N}(\mathbf{x}|\boldsymbol{\mu}_k,\boldsymbol{\Sigma}_k)\]
where $\pi_j\geq 0, \sum_{j=1}^K\pi_j = 1$. (Assume $\mathbf{x},\boldsymbol{\mu}_k\in \mathbb{R}^d,\boldsymbol{\Sigma}_k\in \mathbb{R}^{d\times d}$)

Define the log likelihood as
\[ l(\theta) = \sum_{n=1}^N \log p(\mathbf{x}_n|\theta)
\]
Denote the posterior responsibility that cluster $k$ has for datapoint $n$ as follows:
\[
r_{nk}:=p(z_n=k|\mathbf{x}_n,\theta) = \frac{\pi_k\mathcal{N}(\mathbf{x}_n|\boldsymbol{\mu}_k,\boldsymbol{\Sigma}_k)}{\sum_{k'}\pi_{k'}\mathcal{N}(\mathbf{x}_n|\boldsymbol{\mu}_{k'},\boldsymbol{\Sigma}_{k'})}
\]
\begin{itemize}
	
	\item Show that the gradient of the log-likelihood wrt $\boldsymbol{\mu}_k$ is
	\[ \frac{d}{d\boldsymbol{\mu}_k}l(\theta) = \sum_n r_{nk}\boldsymbol{\Sigma}_k^{-1}(\mathbf{x}_n-\boldsymbol{\mu}_k)
	\]
    \item Derive the gradient of the log-likelihood wrt $\pi_k$ without considering any constraint on $\pi_k$. (bonus 2 points: with constraint $\sum_k\pi_k=1$.)
	
\end{itemize}


\end{document}